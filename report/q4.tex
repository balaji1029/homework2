\section{Quality in Inequalities}

\subsection{Task A}

\begin{align*}
    \mathbb{E}[X] & =\int_{0}^{\infty} X P(X) \, dX \\
    & =\int_{0}^a X P(X) \, dX + \int_{a}^{\infty} X P(X) \, dX \\
    & \ge \int_{0}^{a} X P(X) \, dX + \int_{a}^{\infty} a P(X) \, dX \\
    & \ge \int_{0}^{a} X P(X) \, dX + a P[X \ge a] \\
    & \ge a P[X \ge a] \\
\end{align*}
Here, we have used the fact that $\int_{0}^{a} X P(X) dX$ is $\ge 0$ since $X \ge 0$ and $P(X) \ge 0$ everywhere in that interval.\\
From the above result, we get
\[
    \frac{\mathbb{E}[X]}{a} \ge P[X \ge a]
\]
Intuitively, this corresponds to saying that if there is a high probability that $X \ge a$, then we can also expect that a random sample will give us a value greater than a.
\subsection{Task B}
First, we will prove a result which is useful later
\begin{align*}
    \mathbb{E}[\left(X-\mathbb{E}[X]+\frac{\sigma^2}{\tau}\right)^2]=Var(X-\mathbb{E}[X]+\frac{\sigma^2}{\tau})+\left(E[X-\mathbb{E}[X]+\frac{\sigma^2}{\tau}]\right)^2\\
\end{align*}
Now, since the variance of a distribution does not change on adding a constant to the distribution, and since the expectation value of the sum of terms is the sum of expectations of the terms, we get that this is equal to 
\[
    \mathbb{E}[\left(X-\mathbb{E}[X]+\frac{\sigma^2}{\tau}\right)^2]=\sigma^2+\left(\frac{\sigma^2}{\tau}\right)^2
\]
% \begin{align*}
%     \frac{\mathbb{E}[(X-\mathbb{E}[X]+\frac{\sigma^2}{\tau})^2]}{(\tau+\frac{\sigma^2}{\tau})^2}
% \end{align*}
Now, we notice that whenever $X-\mu \ge \tau$, it also holds that $\left(X-\mu+\frac{\sigma^2}{\tau}\right)^2\ge\left(\tau +\frac{\sigma^2}{\tau}\right)^2$. Since we are assuming that $x-\mu+\frac{\sigma^2}{\tau}\ge \tau+\frac{\sigma^2}{\tau} \ge 0$, this result is proven by the fact that $x^2$ is an increasing function over the non-negative reals.\\
Thus, $P[X-\mu \ge \tau] \le P[\left(X-\mu+\frac{\sigma^2}{\tau}\right)^2\ge\left(\tau +\frac{\sigma^2}{\tau}\right)^2]$.\\
Using Markov's inequality on the above result, we get
\begin{align*}
    P[X-\mu \ge \tau] \le P[\left(X-\mu+\frac{\sigma^2}{\tau}\right)^2&\le\left(\tau +\frac{\sigma^2}{\tau}\right)^2]\le \frac{\sigma^2+\left(\frac{\sigma^2}{\tau}\right)^2}{\left(\tau+\frac{\sigma^2}{\tau}\right)^2}\\
    &=\frac{\sigma^2}{\sigma^2+\tau^2}
\end{align*}
Thus proving our desired result.
\subsection{Task C}
\begin{align*}
    P[X\ge x] = \int_{x}^{\infty} P(X) \, dX
\end{align*}
\begin{align*}
    e^{-tx}E[e^{tx}] & = e^{-tx} \int_{-\infty}^{\infty} e^{tX} P(X) \, dX && \\
    & = e^{-tx}\int_{-\infty}^x e^{tX} P(X) \, dX + e^{-tx}\int_{x}^{\infty} e^{tX} P(X) \, dX && \\
    & \ge e^{-tx} \int_{x}^{\infty} e^{tX} P(X) \, dX && \text{(since both P(X) and $e^{tX}$ are positive for all X)}\\
    & \ge e^{-tx}  e^{tx} \int_{x}^{\infty} P(X) \, dX &&\text{(since $e^{tX}$ is increasing if t>0)}\\
    & = e^{-tx} e^{tx} P[X \ge x] && \\
    & = P[X \ge x] &&
\end{align*}
This proves that 
\begin{equation}
    \label{eq:1}
    P[X\ge x] \le e^{-tx}E[e^{tx}]
\end{equation}
Similarly, 
\[
    P[X\le x] = \int_{-\infty}^{x} P(X) \, dX
\]
\begin{align*}
    e^{-tx}E[e^{tx}] & = e^{-tx} \int_{-\infty}^{\infty} e^{tX} P(X) \, dX && \\
    & = e^{-tx}\int_{-\infty}^x e^{tX} P(X) \, dX + e^{-tx}\int_{x}^{\infty} e^{tX} P(X) \, dX && \\
    & \ge e^{-tx} \int_{-\infty}^{x} e^{tX} P(X) \, dX && \text{(since both P(X) and $e^{tX}$ are positive for all X)}\\
    & \ge e^{-tx}\cdot e^{tx} \cdot \int_{-\infty}^{x}P(X)\, dX && \text{(since $e^{tx} \le e^{tx'}$ where $x \ge x'$ when t is negative)}\\
    & = \int_{-\infty}^{x} P(X) \, dX && \\
    & = P[X \le x] &&
    \label{eq:1}
\end{align*}
as required.
\subsection{Task D}
\begin{enumerate}
    \item We know, by the linearity of expectation, that \[
        \mathbb{E}[Y]=\sum_{i=1}^{n}\mathbb{E}[X_i]=\sum_{i=1}^{n}p_i
    \]
    \item For this question, we make use of the first result proved in Task C above\ref{eq:1}. 
    \begin{align*}       
        \label{Hello}
        P[Y\ge (1+\delta)\mu]&\le \frac{\mathbb{E}[e^{tY}]}{e^{(1+\delta)\mu t}}\\
        &=\frac{\mathbb{E}[e^{\left(t \sum_{i=1}^n X_i\right)}]}{e^{(1+\delta)\mu t}}
    \end{align*}options
    We will now prove that for independent random variables, $\mathbb{E}[e^{\left(t\sum_{i=1}^n X_i\right)}]=\prod_{i=1}^n \mathbb{E}[e^{tX_i}]$. To prove that, we prove the simpler result that for independent variables, $\mathbb{E}[XY]=\mathbb{E}[X]\mathbb{E}[Y]$.
    \begin{align*}
        \mathbb{E}[XY]&=\int_{-\infty}^{\infty}\int_{-\infty}^{\infty}(xy)p_{X,Y}(x,y) \, dy \, dx\\
        & = \int_{-\infty}^{\infty}\int_{-\infty}^{\infty}(xy)p_X(x)p_Y(y) \, dy \, dx && \text{(For independent RVs, p(xy)=p(x)p(y))}\\
        &= \int_{-\infty}^{\infty}x p_X(x) \int_{-\infty}^{\infty}yp_Y(y) \, dy \, dx\\
        &= \int_{-\infty}^{\infty}x p_X(x) \, dx \int_{-\infty}^{\infty}y p_Y(y) \, dy\\
        &= \mathbb{E}[X]\mathbb{E}[Y]
    \end{align*}
    Since above, all $X_i$'s are independent, so are $e^{tX_i}$ for all i. Thus,
    \begin{align*}
        E[e^{\left(t \sum_{i=1}^n X_i\right)}]&=\mathbb{E}[e^{tX_1}]E[e^{\left(t \sum_{i=2}^n X_i\right)}]\\
        &=\mathbb{E}[e^{tX_1}]\mathbb{E}[e^{tX_2}]\mathbb{E}[e^{\left(t \sum_{i=3}^n X_i\right)}]\\
        &\cdots\\
        &=\prod_{i=1}^n \mathbb{E}[e^{tX_i}]
    \end{align*}
    For each Bernoulli random variable $X_i$, it is easily observed that $E[e^{tX_i}]=1+(e^t-1)p_i$. Using this result,
    \[
        E[e^{\left(t \sum_{i=1}^n X_i\right)}]=\prod_{i=1}^n (1+(e^t-1)p_i)
    \]
    We will now prove $\prod_{i=1}^n (1+x_i) \le e^{\sum_{i=1}^n x_i}$ for positive $x_i$'s.
    \begin{align*}
        \ln\left(\prod_{i=1}^n (1+x_i)\right)&=\sum_{i=1}^n \ln(1+x_i)\\
        &\le \sum_{i=1}^n x_i && \text{(Well known result that $\ln(1+x)\le x$)}\\
        &= \ln\left(e^{\sum_{i=1}^n x_i}\right)
    \end{align*}
    Since our initial assumption in using \ref{eq:1} required $t>0$ and hence $e^t > 1$, all $(e^t-1)p_i$ values are positive and we can use the above result.
    Hence, $\mathbb{E}[e^{tY}] \le e^{\mu (e^t-1)}$.\\
    Plugging this into \ref{Hello}, we get
    \[
        P[Y\ge (1+\delta)\mu]\le \frac{e^{\mu(e^t-1)}}{e^{(1+\delta)t\mu}}
    \]
    as required.\\
    % However,  $(1+\delta)^{1+\delta}$ can be simplified by taking its logarithm.
    % \begin{align*}
    %     \ln((1+\delta)^{1+\delta})&=(1+\delta)\ln(1+\delta)\\
    %     &=(1+\delta)(\delta - \delta^2 )
    % \end{align*}
    \item Since this holds for arbitrary positive t, we can find the t for which the value of $e^{\mu (e^t-1-(1+\delta)t)}$ is minimum. By differentiation, we observe the minima of the function is when $t=\ln(1+\delta)$. At the minima, we have
    \begin{align*}
        e^t-1-(1+\delta)t = \delta-(1+\delta)\ln(1+\delta)\\
        % P[Y\ge (1+\delta)\mu]\le \frac{e^{\delta\mu}}{(1+\delta)^{1+\delta}}    
    \end{align*}
    Using the expansion $\ln(1+\delta)=\sum_{i=1}^{\infty}\frac{\delta^i(-1)^{i-1}}{i}$,
    \begin{align*}
        \delta - (1+\delta)\ln(1+\delta) &= \delta + (1+\delta)(\sum_{i=1}^{\infty}\frac{\delta^i(-1)^i}{i})\\
        &= \delta + \sum_{i=1}^{\infty}\frac{\delta^i(-1)^i}{i} + \sum_{i=2}^{\infty}\frac{\delta^{i}(-1)^{i-1}}{i-1}\\
        &= \sum_{i=2}^{\infty} \frac{\delta^i(-1)^{i-1}}{i(i-1)}\\
        % &= -\delta^2 \mu \sum_{i=0}^{\infty}\frac{(\delta^i)(-1)^i}{(i+2)(i+1)}
    \end{align*}
    This is approximately $-\frac{\delta^2}{2} + \frac{\delta^3}{6}$ for small delta. Thus we get a good upper bound of $e^{\frac{-\mu\delta^2(3-\delta)}{6}}$ for $P[Y\ge (1+\delta)\mu]$.\\
    If $\delta > 0$, this can be simplified further, in a way that is more useful for us in question 5. Observe that for $x > 0$
    \[
        \ln(1+x) \ge \frac{2x}{2+x}
    \]
    This can be seen by observing that at x=0, $\ln(1+x)-\frac{2x}{2+x}$ evaluates to 0 and that its derivative, $\frac{4}{(1+x)(2+x)^2}$ is always non-negative.\\
    Thus, 
    \begin{align*}
        \mu (\delta - (1+\delta)\ln(1+\delta)) &\le \mu (\delta - \frac{2\delta(1+\delta)}{2+\delta})\\
        &= \frac{-\mu\delta^2}{2+\delta}
    \end{align*}
    Thus, we get,
    \begin{equation}
        \label{eq:2}
        P[Y\ge (1+\delta)\mu] \le e^{\frac{-\mu\delta^2}{2+\delta}}
    \end{equation}
    For the last few steps of simplification, https://math.mit.edu/~goemans/18310S15/chernoff-notes.pdf was used as a guide.
\end{enumerate}
\subsection{Task E}
First, let us consider the corner cases where $\mu = 0$ or $\mu = 1$. In these cases, $A_n=0$ or $A_n=1$ always, and the result is trivial.\\ 
Now we can consider $0 < \mu < 1$.\\
Consider the random variable $B_n = nA_n$.\\
We need to prove $\lim_{n\to \infty}P[\lvert B_n - n\mu\rvert > n\epsilon]=0$.\\
Note that we are done if we prove $\lim_{n\to \infty}P[B_n - n\mu >= n\epsilon]=0$ and $\lim_{n\to \infty}P[n\mu - B_n >= n\epsilon]=0$.\\
We will prove the first part first, assuming $\mu$ is not 0.\\
\begin{align*}
    P[B_n \ge n\epsilon + n\mu] &\le e^{-\frac{n\epsilon^2}{2\mu+\epsilon}}\\
\end{align*}
For a positive $\mu$ (since the distribution is the sum of Bernoulli RVs) and $\epsilon$ (given), $\frac{\epsilon^2}{2\mu+\epsilon}$ is a positive coefficient. $e^{-n}$ tends to 0 as n tends to $\infty$, which we will use henceforth.\\
Note that the result that $P[X \ge (1+\delta)\mu] \le e^{\frac{-\mu\delta^2}{2+\delta}}$ was not proven without referring to the internet, so I will prove the claim above using an alternative method which assumes only what has been proven by me.\\
\begin{align*}
    P[X \ge (1+\delta)\mu] \le e^{\mu(\delta-(1+\delta)\ln(1+\delta))}
\end{align*}
We want to prove that $f(\delta)=\delta-(1+\delta)\ln(1+\delta)$ is strictly negative if $\delta$ is positive. For that, notice that $f(0)=0$ and $f'(\delta)=-\ln(1+\delta)$, which is non-positive for all non-negative $\delta$. Thus, $f(\delta)$ is strictly negative for all positive $\delta$.\\
This again means that $P[B_n \ge n\epsilon + n\mu] \le e^{-nk}$ for some positive coefficient k. Thus, $P[B_n \ge n\epsilon + n\mu]$ tends to 0 as n tends to $\infty$.\\
Similarly, $P[B_n \le n\mu - n\epsilon]$ also tends to 0 as n tends to $\infty$.\\
We can prove this by considering the RV $1-A_n$ instead of $A_n$ and using the same logic as above. $P[n-B_n - n-n\mu \ge n\delta] = P[n\mu - nB_n \ge n\delta] \le e^{-nk}$ for some positive coefficient k.\\
Now, combining the two results,
\begin{align*}
    P[\lvert B_n - n\mu\rvert > n\epsilon] &\le P[B_n - n\mu \ge n\epsilon] + P[n\mu - B_n \ge n\epsilon]\\
    &\le e^{-nk} + e^{-nk'} && \text{(for some positive coefficients k and k')}\\
\end{align*}
If we get any positive value $\sigma$ for the value of $P[\lvert B_n - n\mu\rvert > n\epsilon]$, simply set $n > \max(\frac{\ln(\frac{2}{\sigma})}{k}, \frac{\ln(\frac{2}{\sigma})}{k'})$ and we have
\begin{align*}
    P[\lvert B_n - n\mu\rvert > n\epsilon] &\le e^{-nk} + e^{-nk'}\\
    &< \frac{\sigma}{2} + \frac{\sigma}{2}\\
    &= \sigma
\end{align*}
which is a contradiction.\\
Thus, we have proven that $\lim_{n\to \infty}P[\lvert B_n - n\mu\rvert > n\epsilon]=\lim_{n\to \infty}P[\lvert A_n - \mu\rvert > \epsilon]=0$.\\